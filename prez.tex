\documentclass{beamer}
\usepackage[polish]{babel}
\usepackage[utf8]{inputenc}
\usepackage[T1]{fontenc}
\usepackage[colorlinks=true, linkcolor=blue, urlcolor=cyan]{hyperref}
\usepackage{graphicx}

\title{ChadGuide}
\subtitle{Planowanie podróży po Europie}
\author{Paweł Kauf, Filip Rabiega}
\date{\today}

\begin{document}
\maketitle
\begin{frame}
  \frametitle{Wstęp}
  \emph{Problem}: Planowanie podróży wielomiastowej wymaga
  sprawdzenia wykładniczej liczby kombinacji kolejności miast. Ręczne
  porównywanie jest niepraktyczne.

  \emph{Cel}: Znalezienie optymalnych (najtańszych/najkrótszych) tras
  lotniczych łączących wiele miast europejskich.
\end{frame}
\begin{frame}{Wstęp}
  \emph{ChadGuide} jest \emph{ChadRozwiązaniem}.
  \begin{itemize}
    \item Oprócz optymalizacji wykonuje jeszcze walidacje lotów
    \item Wszystko dzieje się przez wygodną oraz przejrzystą aplikację webową
  \end{itemize}
\end{frame}
\begin{frame}{Użyte technologie}
  \begin{itemize}
    \item Python: \texttt{poetry}, \texttt{pandas},
      \texttt{streamlit}, \texttt{pandera}, \texttt{fastapi},
      \texttt{uvicorn}, \texttt{pydantic}, \texttt{sse-starlette},
      \texttt{playwright}, \texttt{pytest}
    \item JavaScript
    \item SQLite
    \item API: Duffel, Gepapify
  \end{itemize}
\end{frame}
\begin{frame}{Demo}
  \begin{itemize}
    \item Nasz projekt można obejrzeć na
      \href{https://github.com/fiadra/chadguide}{https://github.com/fiadra/chadguide}.
    \item Działającą publiczną instancją jest
      \href{https://www.chadguide.site}{https://www.chadguide.site}.
  \end{itemize}
\end{frame}
\begin{frame}{Zbieranie danych o lotach}
  \begin{itemize}
    \item Zbieranie danych o lotach do bazy jest realizowane przez
      paczkę \texttt{flight scanner}, która łączy się z DuffelAPI
      (148 lotnisk, 89k lotów)
    \item Dane w naszej bazie zawsze będą trochę do tyłu względem
      prawdziwych, ponieważ koszty biletów oraz ich dostępność ciągle
      się zmieniają. Dane dla tygodnia
      01.07.2026--07.07.2026 są ekstrapolowane na cały miesiąc.
  \end{itemize}
\end{frame}
\begin{frame}{Zbieranie danych o atrakcjach}
  \begin{itemize}
    \item Zbieranie danych o atrakcjach dzieje się asynchronicznie
      podczas działania algorytmu. Służy do tego paczka, która
      odpytuje Geoapify o dane.
    \item Zdjęcia atrakcji zostały zebrane metodami webscrapingu.
  \end{itemize}
\end{frame}
\begin{frame}{Algorytm}
  \begin{itemize}
    \item Sercem projektu jest zmodyfikowany algorytm Dijkstry, który
      wyznacza drogi Pareto-optymalne względem ceny oraz czasu
      trwania lotu. Jednocześnie radzi sobie on z grafami, które są
      zmienne w czasie.
    \item Droga \( p \) \emph{dominuje} drogę \( q \), gdy istnieje
      kryterium, w którym \( p \) jest lepsze od \( q \) oraz nie
      istnieje kryterium, w którym \( p \) jest gorsze niż \( q \).
    \item Droga jest \emph{Pareto optymalna}, gdy nie jest dominowana
      przez żadną inną drogę.
  \end{itemize}
\end{frame}
\begin{frame}{Algorytm}
  \begin{itemize}
    \item Interesują nas tylko te drogi, które wychodzą i wracają z
      wyznaczonego miasta, oraz przelatują przez wszystkie miasta,
      które użytkownik chce odwiedzić.
    \item Inne drogi są automatycznie odrzucane.
    \item Przed uruchomieniem algorytmu wykonujemy prosty
      \emph{pruning}, tzn. pozbywamy się tych połączeń w grafie,
      które prawie na pewno nie będą częścią optymalnego rozwiązania.
  \end{itemize}
\end{frame}
\begin{frame}{Algorytm}
  \includegraphics[width=1\textwidth]{alg.png}
\end{frame}
\begin{frame}{Walidacja}
  \begin{itemize}
    \item Znalezione drogi są \emph{walidowane}, tzn. sprawdzamy, czy
      faktycznie istnieją.
    \item Top 5 tras jest weryfikowanych asynchronicznie z API w
      czasie rzeczywistym.
  \end{itemize}
\end{frame}
\begin{frame}{Diagram}
  \includegraphics[width=1\textwidth]{diag.png}
\end{frame}
\begin{frame}{Git, dokumentacja, testy, ...}
  \begin{itemize}
    \item Całość powstała dzięki systemowi kontroli wersji Git.
    \item Staraliśmy się prowadzić dokładną dokumentację
      poszczególnych modułów oraz paczek.
    \item Do wszystkich istotnych części projektu napisaliśmy testy,
      również te wydajnościowe. Do testów została użyta paczka \texttt{pytest}.
  \end{itemize}
\end{frame}
\begin{frame}{Co mogliśmy zrobić lepiej?}
  \begin{itemize}
    \item Za dużo czasu zajęło nam znalezienie odpowiedniego dostawcy
      danych lotów.
    \item blabla
  \end{itemize}
\end{frame}
\begin{frame}{Podsumowanie}

\end{frame}
\end{document}
