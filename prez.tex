\documentclass{beamer}
\usepackage[polish]{babel}
\usepackage[utf8]{inputenc}
\usepackage[T1]{fontenc}
\usepackage[colorlinks=true, linkcolor=blue, urlcolor=cyan]{hyperref}

\title{ChadGuide}
\subtitle{Planowanie podróży po Europie}
\author{Paweł Kauf, Filip Rabiega}
\date{\today}

\begin{document}
\maketitle
\begin{frame}
  \frametitle{Wstęp}
  \emph{Problem}: Planowanie podróży wielomiastowej wymaga
  sprawdzenia wykładniczej liczby kombinacji kolejności miast. Ręczne
  porównywanie jest niepraktyczne.

  \emph{Cel}: Znalezienie optymalnych (najtańszych/najkrótszych) tras
  lotniczych łączących wiele miast europejskich.
\end{frame}
\begin{frame}
  \frametitle{Użyte technologie}

\end{frame}
\begin{frame}
  \frametitle{Demo}
  \begin{itemize}
    \item Nasz projekt można obejrzeć na
      \href{https://github.com/fiadra/chadguide}{https://github.com/fiadra/chadguide}.
    \item Działającą publiczną instancją jest
      \href{https://www.chadguide.site}{https://www.chadguide.site}.
  \end{itemize}
\end{frame}
\begin{frame}
  \frametitle{Algorytm}
  \begin{itemize}
    \item Sercem projektu jest zmodyfikowany algorytm Dijkstry, który
      wyznacza drogi Pareto-optymalne względem ceny oraz czasu trwania lotu.
    \item Droga \( p \) \emph{dominuje} drogę \( q \), gdy istnieje
      kryterium, w którym \( p \) jest lepsze od \( q \) oraz nie
      istnieje kryterium, w którym \( p \) jest gorsze niż \( q \).
    \item Droga jest \emph{Pareto optymalna}, gdy nie jest dominowana
      przez żadną inną drogę.
  \end{itemize}
\end{frame}
\begin{frame}
  \frametitle{Algorytm}
  \begin{itemize}
    \item Interesują nas tylko te drogi, które wychodzą i wracają z
      wyznaczonego miasta, oraz przelatują przez wszystkie miasta,
      które użytkownik chce odwiedzić.
    \item Inne drogi są automatycznie odrzucane.
  \end{itemize}
\end{frame}
\end{document}
